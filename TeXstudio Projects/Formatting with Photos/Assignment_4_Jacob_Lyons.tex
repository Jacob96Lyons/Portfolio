\documentclass{article}
\usepackage{amsmath, amsfonts, amsthm, amssymb}
\usepackage[utf8]{inputenc}
\usepackage[margin=2cm,heightrounded=true,centering]{geometry}%eraseifyouwantlargemargin
\usepackage{graphicx}
\usepackage{hyperref}

\title{The Origin of Zero}
\author{Jacob Lyons}
\date{7/5/2021}

\newtheorem{lastname}{Lyons}[section]
\newtheorem{firstname}{Jacob}
\newcounter{newc}
\hypersetup{colorlinks=true, linkcolor=blue, filecolor=magenta, urlcolor=cyan,}

\newenvironment{new environment}[1]
{\begin{center}
		#1\vspace{0.1in}\\
		\begin{tabular}{|p{0.9\textwidth}|}
			\hline\\
		}
		{ 
			\\\\\hline
		\end{tabular} 
	\end{center}
}

\newcounter{ex}[section]
\newenvironment{counter environment}[1]
{
	{\refstepcounter{ex}}
	\begin{center}
		#1\vspace{0.1in}\\
		\begin{tabular}{|p{0.9\textwidth}|}
			\hline\\
		}
		{ 
			\\\\\hline
		\end{tabular} 
	\end{center}
}


\newenvironment{last environment}[1]
{
	{\refstepcounter{ex}}
	\begin{center}
		\textbf{New envirionment \theex.}
		#1\vspace{0.1in}\\
		\begin{tabular}{|p{0.9\textwidth}|}
			\hline\\
		}
		{ 
			\\\\\hline
		\end{tabular} 
	\end{center}
}


\begin{document}
	
	\maketitle
	
	\section{Figures}
	The first image should be centered and the size of it shouldn’t be wider than 0.5 textwidth. The image should
	contain a caption and it should be labeled:	\\
	
	\begin{figure}[h] %%%h,t,p is an option that I’ll explain soon.
		\centering %%%should be clear what this does.
		\includegraphics[width=0.3\textwidth]{tex} %%I used 0.3 because it matched the document given, otherwise i would've used 0.5%%
		\caption{\label{latex}first image}
	\end{figure}

\noindent Once the image is labeled, I can refer to it using \textbackslash ref\{latex\} like this: \ref{latex}	\\

\noindent The second image should be centered and the size of it should be wider than 0.5 textwidth. It should also be placed in a new page with a labeled caption, \ref{latex2}

	
	\begin{figure}[p] %%%h,t,p is an option that I’ll explain soon.
		\centering %%%should be clear what this does.
		\includegraphics[width=0.5\textwidth]{tex}
		\caption{\label{latex2}second image}	
	\end{figure} 



	\section{Environment}	
	Define a new environment that does this:
		\begin{last environment}{This is where the argument goes.}
			\begin{center}
				This is a new numbered environment!
			\end{center}
		\end{last environment}

	\noindent This is a numbered environment.
		\begin{last environment}{This is where the argument goes.}
			\begin{center}
				So this is New environment 2!
			\end{center}
		\end{last environment}

\newpage	
	
	\section{environment2}
		\begin{last environment}{This is where the argument goes.}
			\begin{center}
				And it must reset the counter when you start a new section.
			\end{center}
		\end{last environment}



\end{document}

